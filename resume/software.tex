\cvsection{Software Development}

\begin{cventries}
  \cventry
    {\href{https://github.com/sharppy/SHARPpy}{Sounding and Hodograph Analysis and Research Package in Python}} % Empty position
    {SHARPpy} % Project
    {} % Empty location
    {April 2014 -- Current} % Empty date
    {
      \begin{cvitems} % Description(s) bullet points
        \item {SHARPpy provides an open source source of sounding and hodograph analysis routines to the meteorological community and is used internationally.  The package has been has been referenced over 70 times in peer-reviewed literature.  Development involves collaborations with NWS/SPC employees and is co-developed with Kelton Halbert (University of Wisconsin-Madison) and Tim Supinie (OU/CAPS).}
      \end{cvitems}
    }
  \cventry
    {\href{https://github.com/sharppy/SHARPpy}{Optimal estimation retrievals for ground-based microwave and IR sensors}} % Empty position
    {MWRoe/AERIoe} % Project
    {} % Empty location
    {April 2013 -- May 2018} % Empty date
    {
      \begin{cvitems} % Description(s) bullet points
        \item {MWRoe/AERIoe are retrieval algorithms using the optimal estimation method to solve for the thermodynamic profiles, cloud properties, and trace gas concentrations from atmospheric spectral IR and microwave observations.  Developed in conjunction with Dave Turner (NOAA/ESRL) in Turner and Blumberg (2019) and is succeeded by the TROPoe software package.}
      \end{cvitems}
    }
    
   \cvskill
     {Languages:}
     {Python, IDL, FORTRAN, C++, C}
     
\end{cventries}