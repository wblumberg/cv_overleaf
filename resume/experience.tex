\cvsection{Research Experience}

\begin{cventries}
  \cventry
    {NASA Postdoctoral Program Fellow} % Job title
    {NASA Goddard Space Flight Center} % Organization
    {Greenbelt, MD} % Location
    {July 2019 - Now} % Date(s)
    {
      \begin{cvitems} % Description(s) of tasks/responsibilities
        \item {Studied atmospheric boundary layer and aerosol feedbacks from biomass burning in Southeast Asia using small Uncrewed Aerial Systems (sUAS), ground-based profilers, and the NASA Unified WRF (NU-WRF) system.  Collaborated with scientists at National Central University in Taiwan and National Oceanic and Atmospheric Administration as a part of NASA's Seven SouthEast Asian Studies (7-SEAS) Mission.}
        \item {Collaborated with the range safety community to study the meteorological effects on the distant focusing overpressure risk at coastal launch ranges and explore future methods of assessing this risk.}
        \item {Supported colleagues with boundary layer expertise to advance incubation activities at NASA Goddard for the Planetary Boundary Layer 2017 Decadal Survey goals.}
      \end{cvitems}
    }

  \cventry
    {Graduate Research Assistant} % Job title
    {Cooperative Institute for Mesoscale Meteorology Studies} % Organization
    {Norman, OK} % Location
    {June 2014 -- May 2018} % Date(s)
    {
      \begin{cvitems} % Description(s) of tasks/responsibilities
        \item {Assisted in developing ground-based observation strategies for the \href{https://www.eol.ucar.edu/field_projects/pecan}{Plains Elevated Convection At Night} (PECAN) field project.  Also participated in deployments of the \href{http://www.nssl.noaa.gov/users/dturner/public_html/CLAMPS/}{Collaborative Lower Atmosphere Mobile Profiling System} (CLAMPS).}
        \item {Performed research designed to improve the understanding of how radiosondes and ground based remote sensors (e.g., AERI, Doppler wind lidar, microwave radiometer) instruments can be used to diagnose the ingredients relevant to deep moist convection.}
        \item {Designed and performed experiments using high-temporal resolution, ground-based instrumentation to improve our understanding of the processes that drive the rapid evolution of instability, moisture, and shear during the Southern Great Plains afternoon to evening transition.}
      \end{cvitems}
    }

  \cventry
    {Research Fellow} % Job title
    {} % Organization
    {} % Location
    {August 2013 -- June 2014} % Date(s)
    {
      \begin{cvitems} % Description(s) of tasks/responsibilities
        \item {Developed ways to enhance the speed and accuracy of the AERI optimal estimation (AERIoe) thermodynamic retrieval algorithm.}
        \item {Evaluated the accuracy of AERI retrieval methods in comparison to other thermodynamic profiling technologies.}
        \item {Mentored student research on AERI retrievals to assess the accuracy of planetary boundary layer parameterization schemes from convective-scale ensemble numerical weather prediction forecasts.$^{1}$}
      \end{cvitems}
    }

  \cventry
    {Graduate Research Assistant} % Job title
    {University of Oklahoma School of Meteorology} % Organization
    {Norman, OK} % Location
    {January 2012 -- August 2013} % Date(s)
    {
      \begin{cvitems} % Description(s) of tasks/responsibilities
        \item {Developed and tested statistical thermodynamic retrievals for the AERI, a ground-based passive infrared remote sensing system, by using linear regression, principle component analysis, and neural networks.}
		\item {Provided expertise on AERI thermodynamic retrievals to study nocturnal low-level jet streams during the student run, multi-institutional Lower Atmospheric Boundary Layer Experiment (LABLE) field experiment.}
      \end{cvitems}
    }

  \cventry
    {Undergraduate Research Intern} % Job title
    {NOAA Storm Prediction Center} % Organization
    {Norman, OK} % Location
    {Fall 2010} % Date(s)
    {
      \begin{cvitems} % Description(s) of tasks/responsibilities
        \item {Performed interviews with National Weather Service Warning Coordination Meteorologists to document common severe weather preparedness criteria used by StormReady certified amusement parks and developed new criteria to be applied to future amusement park StormReady certification.}
		\item {Developed a database of U.S. large venue locations to be integrated into AWIPS for the purpose of heightening forecaster situational awareness.}
      \end{cvitems}
    }

\end{cventries}
